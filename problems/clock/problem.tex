\begin{problem}{고장난 시계}{standard input}{standard output}

논산훈련소에 간 불쌍한 상언이는 첫날 훈련소에서 다쳐 먼 거리를 이동할 때 버스를 타고 다녔다. 어느 날 버스를 탄 상언이는 훈련소 버스 앞에 붙어있는 시계를 보게 되었는데, 시계가 이상해 보인다는 사실을 관찰했다. 일반적인 시계라면 가리킬 수 없는 시간을 가리키고 있던 것이다. 이 시계를 본 상언이는 어떤 시계의 시침과 분침이 가리키는 방향을 보면 그 시계가 고장 났는지 정상인지 판단할 수 있을 거라 생각했지만, 귀찮아서 생각을 그만두기로 했다. 상언이를 대신해서 시계가 정상인지 이상한지 알려주자.


\InputFile
첫 번째 줄에 시침의 각도 $\theta_1$, 분침의 각도 $\theta_2$ ($0 \le \theta_1, \theta_2 \le 359$)가 정수로 주어진다.
시침, 분침의 각도는 12시 방향을 기준으로 시계방향으로 잰다. 예를 들어 3시 방향은 90, 9시 방향은 270 이다.

\OutputFile
첫 번째 줄에 시계의 각도가 정상인 경우 \texttt{O}를, 그렇지 않을 경우 \texttt{X}를 출력한다.

\Example

\begin{example}
\exmp{180 0}{O}%
\exmp{0 180}{X}%
\end{example}

\Notes
첫 번째 예시는 시침이 6시 방향을, 분침이 12시 방향을 가리키고 있는 상태로 6시 정각의 시계 모양이다.

두 번째 예시는 시침이 12시 방향을, 분침이 6시 방향을 가리키고 있는 상태로 정상적인 시계에선 나올 수 없는 모양이다.

\end{problem}
