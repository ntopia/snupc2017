\begin{problem}{구간 합 최대}{standard input}{standard output}

승현이는 음이 아닌 정수 $N$개로 구성된 배열을 가지고 놀고 있다. 이 배열에는 $M$개의 특별한 조건이 있는데, 그것은 길이가 $L_i$인 연속한 구간을 잡았을 때 합이 $S_i$를 넘지 않는다는 것이다. 특별한 조건을 모두 만족하는 배열은 여러 가지일 수 있는데, 길이가 $K$인 연속한 구간둘 중 합이 가장 큰 것이 있을 것이다. $1$ 이상 $N$ 이하의 모든 정수 $K$에 대해서, 특별한 조건을 모두 만족하는 배열의 길이가 $K$인 연속한 구간들 중 합이 가장 큰 구간의 구간 합을 구하여라.

\InputFile
첫 번째 줄에 배열을 구성하는 정수의 개수 $N$($1 \le N \le 200,000$)과 특별한 조건의 개수 $M$($1 \le M \le 200$)이 주어진다.

두 번째 줄부터 $M$개의 줄에 걸쳐 특별한 조건을 의미하는 두 정수 $L_i$와 $S_i$가 주어진다. ($1 \le i \le M, 1 \le L_i \le N, 1 \le S_i \le 10^9$)

\OutputFile
$N$줄에 걸쳐 $K$번째 줄에 특별한 조건을 모두 만족하는 배열의 길이가 $K$인 연속한 구간들 중 합이 가장 큰 구간의 구간 합을 출력한다.

\Example

\begin{example}
\exmp{5 2
2 5
3 7}{5
5
7
10
12}%
\end{example}

\Notes

배열이 [1, 4, 1, 0, 5]일 때, 길이가 1인 연속한 구간들 중 합이 5인 것이 존재하고, 길이가 2인 연속한 구간들 중 합이 5인 것이 존재하고, 길이가 4인 연속한 구간들 중 합이 10인 것이 존재한다.

배열이 [3, 2, 2, 1, 4]일 때, 길이가 3인 연속한 구간들 중 합이 7인 것이 존재하고, 길이가 5인 연속한 구간들 중 합이 12인 것이 존재한다.

\end{problem}
