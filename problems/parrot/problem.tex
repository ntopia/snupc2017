\begin{problem}{앵무새}{standard input}{standard output}

자가용 비행기를 타고 세계 일주를 하던 pps789와 cseteram은 어느 날 엔진 고장으로 인해 이름 모를 섬에 불시착하게 된다. 그들은 이 섬을 탐험하는 도중 아주 신기한 사실을 알게 되었는데, 바로 이 섬에 사는 앵무새들은 놀라울 정도로 인간의 말을 흉내내는 데 뛰어나다는 것이다. 그들은 서로 떨어져 섬을 탐험하기로 하였으며, 필요하다면 앵무새를 이용해 서로에게 연락을 하기로 약속하였다.

1개월 후, pps789는 섬의 비밀을 밝힐 결정적인 증거를 찾게 된다. 그는 이 세기의 대발견을 cseteram에게 공유하고자 하였으나, 그의 발견은 방대하여 앵무새 한 마리가 기억하기에는 너무 많은 양이었다. 그렇기에 pps789는 앵무새 한 마리 대신 앵무새 $N$마리를 이용하여 자신의 발견을 기록하였으며, 이 앵무새들을 cseteram을 향해 날렸다.

한편 섬의 반대편에서 탐험을 계속하던 cseteram은 앵무새 $N$마리가 자신에게 날아와 각자 할 말을 하는 것을 보고 당황하였다. pps789가 긴 글을 전달하고 싶었던 것은 알아차렸지만, 각각의 앵무새들이 말하는 것을 차례대로 기록하다 보니 원문이 무엇인지 알 수 없을 정도로 단어의 순서가 엉켜버린 것이다. 대신 그는 관찰을 통해 몇 가지 규칙을 발견할 수 있었다.

\begin{enumerate}
\item{한 앵무새는 한 문장을 기억하고 있다. 문장은 여러 단어로 이루어져있는데, 앵무새는 이 단어들을 순서대로 말한다.}
\item{한 앵무새가 단어를 말하고 그 다음 단어를 말하기 전에는 약간의 간격이 있는데, 이 때 다른 앵무새가 말을 가로채고 자신의 문장을 말할 수 있다.}
\item{한 앵무새가 단어를 말하는 도중에는, 다른 앵무새가 말을 가로채지 않는다.}
\end{enumerate}

pps789가 각각의 앵무새들에게 전달한 문장 $S_i$와, cseteram이 받아 적은 문장 $L$이 주어진다. 이 때 문장 $L$이 위 규칙들을 이용하여 나올 수 있는 문장인지 판별하시오.

\InputFile
첫 번째 줄에 앵무새의 수 $N$ ($1 \le N \le 1000$) 이 주어진다.

두 번째 줄부터 $N$개의 줄에 걸쳐 각 앵무새가 말한 문장 $S_i$ ($1 \le i \le N$) 가 주어지는데, 각 문장을 이루는 단어는 스페이스 한 칸을 구분으로 하여 주어진다. 문장 $S_i$를 이루는 단어의 수는 $100$개를 넘지 않으며, 각 단어는 최대 $32$개의 영문 소문자로 구성되어있다.

$N+2$ 번째 줄에는 cseteram이 받아 적은 문장 $L$이 주어진다.

\OutputFile
문장 $L$이 가능한 문장이라면 Possible을, 불가능한 문장이라면 Impossible을 출력한다.

\Example

\begin{example}
\exmp{3
i want to see you
next week
good luck
i want next good luck week to see you}{Possible}%
\exmp{2
i found
an interesting cave
i found an cave interesting}{Impossible}%
\exmp{2
please
be careful
pen pineapple apple pen}{Impossible}%
\end{example}

\end{problem}
