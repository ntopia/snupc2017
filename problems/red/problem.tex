\begin{problem}{홍삼 게임(Easy)}{standard input}{standard output}

은하는 술과 게임과 술 게임을 좋아한다. 그 중에서도 가장 좋아하는 술 게임은 ``홍삼 게임''이다. 이 게임은 $N$명의 사람이 테이블에 원형으로 둘러 앉은 상태에서 하는 게임이고, 규칙은 다음과 같다.

\begin{enumerate}
\item{은하가 서로 다른 두 사람을 지목한다.}
\item{지목당한 두 사람이 동시에 테이블에 앉은 사람들 중 한 사람씩을 골라서 지목한다.}
\item{만약 두 사람이 같은 사람을 지목했을 경우 게임이 끝난다. 그렇지 않을 경우 2번으로 돌아간다.}
\end{enumerate}

SNUPC가 끝난 뒤 참가자들은 근처 술집으로 뒤풀이를 하러 갔고, 은하의 주도 아래 홍삼 게임을 하게 되었다. 하지만 사람이 너무 많이 모이는 바람에 누가 누굴 지목하는지 잘 보이지 않아서 게임이 수시로 중단되었다. 이 상황을 보다 못한 은하의 친구 은서는 홍삼 게임의 룰을 수정한 ``질서 있는 홍삼 게임''을 제안하였다. 이 게임의 규칙은 다음과 같다.

\begin{enumerate}
\item{은하가 서로 다른 두 사람을 순서대로 지목한다. 먼저 지목당한 사람은 지목권 A, 두 번째로 지목당한 사람은 지목권 B를 갖는다.}
\item{지목권 A를 가진 사람이 자신의 왼쪽 또는 오른쪽으로 정확히 $D_{A}$만큼 떨어진 사람 한 명을 지목하여 자신의 지목권을 넘긴다.}
\item{만약 지목당한 사람이 이미 지목권 B를 가지고 있었을 경우 게임이 끝난다.}
\item{지목권 B를 가진 사람이 자신의 왼쪽 또는 오른쪽으로 정확히 $D_{B}$만큼 떨어진 사람 한 명을 지목하여 자신의 지목권을 넘긴다.}
\item{만약 지목당한 사람이 이미 지목권 A를 가지고 있었을 경우 게임이 끝난다. 그렇지 않을 경우 2번으로 돌아간다.}
\end{enumerate}

은서의 제안 덕분에 참가자들은 질서 있게 홍삼 게임을 즐길 수 있게 되었다. 하지만 은하가 몇 시간 내내 계속 홍삼 게임을 돌리자 참가자들은 지쳐 갔고, 은하가 누구를 지목하고 지목 간격을 어떻게 설정하든 최대한 게임을 빠르게 끝내려고 하게 되었다. 불쌍한 뒤풀이 참가자들을 홍삼 지옥에서 구해 주자.

편의를 위해 참가자들에게는 $1$번부터 $N$번까지 반시계 방향으로 번호가 붙어 있다고 가정한다. 즉 $i$번 참가자의 바로 왼쪽에는 $i-1$번, 바로 오른쪽에는 $i+1$번 참가자가 앉아 있다. 예외로 $1$번 참가자의 바로 왼쪽에는 $N$번 참가자가, 마찬가지로 $N$번 참가자의 바로 오른쪽에는 $1$번 참가자가 앉아 있다.

\InputFile
첫 번째 줄에 ``질서 있는 홍삼 게임''의 참가자의 수 $N$($2 \le N \le 500$), 은하가 먼저 지목한 사람의 번호 $A$와 두 번째로 지목한 사람의 번호 $B$($1 \le A, B \le N$, $A \neq B$), 각 지목권의 지목 간격을 나타내는 정수 $D_{A}$, $D_{B}$($1 \le D_{A}, D_{B} \le N-1$)이 공백을 사이에 두고 순서대로 주어진다.

\OutputFile
첫 번째 줄에 입력된 게임을 최대한 빠르게 끝내고자 할 때 필요한 최소 지목 횟수를 출력한다. 만약 끝낼 수 없는 게임일 경우 \texttt{Evil Galazy}를 출력한다.

\Example

\begin{example}
\exmp{6 5 1 1 2}{3}%
\exmp{4 1 2 2 2}{Evil Galazy}%
\end{example}

\Notes

첫 번째 예시는 다음과 같은 순서로 진행하면 3번의 지목으로 끝낼 수 있다.

\begin{enumerate}
\item{지목권 A를 가진 5번 참가자는 4번 또는 6번 참가자를 지목할 수 있다. 이 중 4번 참가자를 지목하여 지목권을 넘긴다.}
\item{지목권 B를 가진 1번 참가자는 5번 또는 3번 참가자를 지목할 수 있다. 이 중 3번 참가자를 지목하여 지목권을 넘긴다.}
\item{지목권 A를 가진 4번 참가자가 3번 참가자를 지목하여 지목권을 넘기고 게임이 끝난다.}
\end{enumerate}

\end{problem}
