\begin{problem}{관악산 등산}{standard input}{standard output}

서울대학교에는 ``누가 조국의 미래를 묻거든 고개를 들어 관악을 보게 하라''라는 유명한 문구가 있다. 어느 날 Unused는 Corea에게 조국의 미래를 물었고, Corea는 직접 관악산에 올라가 조국의 미래를 보고 답해 주기로 했다.

관악산의 등산로는 $1$부터 $N$까지의 서로 다른 번호가 붙어 있는 $N$개의 쉼터와 두 쉼터 사이를 오갈 수 있는 $M$개의 길들로 이루어져 있다. Corea는 지면에서부터 산을 오르는 것은 너무 귀찮다고 생각했기 때문에, 케이블카를 타고 임의의 쉼터에서 내린 다음 등산을 시작하기로 했다. Corea는 항상 더 높은 곳을 지향하기 때문에, 쉼터에 도착하면 그 쉼터와 직접 연결된 더 높은 쉼터로 향하는 길들 중 하나를 골라서 그 길을 따라 이동한다. 만약 그런 길이 없다면 등산을 마친다.

관악산의 쉼터들에는 조국의 미래를 볼 수 있는 전망대가 하나씩 설치되어 있다. Corea는 최대한 많은 쉼터를 방문해서 조국의 미래를 많이 보고 Unused에게 전해 주기로 했다. 관악산의 지도가 주어질 때, Corea가 각각의 쉼터에서 출발해서 산을 오를 때 최대 몇 개의 쉼터를 방문할 수 있는지 구하여라.

\InputFile
첫 번째 줄에 등산로에 있는 쉼터의 수 $N$($1 \le N \le 5,000$)과 두 쉼터 사이를 연결하는 길의 수 $M$($1 \le M \le 100,000$)이 주어진다.

두 번째 줄에는 각 쉼터의 높이를 나타내는 $N$개의 정수가 번호 순서대로 주어진다. 각 쉼터의 높이는 $1$ 이상 $1,000,000$ 이하이며 서로 다르다.

세 번째 줄부터 $M$개의 줄에 걸쳐 각각의 길이 연결하는 두 쉼터의 번호가 공백으로 구분되어 주어진다. 쉼터의 번호는 $1$ 이상 $N$ 이하의 정수이다. 양 끝점이 같은 쉼터인 길은 없으며, 임의의 두 쉼터를 연결하는 길이 여러 개 존재할 수 있다.

\OutputFile

$N$개의 줄에 걸쳐 $n$번째 줄에 Corea가 $n$번 쉼터에서 출발해서 산을 오를 때 최대로 방문할 수 있는 쉼터의 개수를 출력한다.

\Example

\begin{example}
\exmp{5 5
3 1 6 4 7
1 4
2 1
3 4
4 2
5 1}{3
4
1
2
1}%
\end{example}

\Notes
\begin{center}
  \includegraphics[width=0.4\textwidth]{climb.png}
\end{center}

2번 쉼터에서 출발하면 1번, 4번, 3번 쉼터를 차례대로 방문할 때 가장 많은 쉼터를 방문할 수 있다.

5번 쉼터는 3번 쉼터보다 높은 곳에 있지만 길 하나로 연결되어 있지 않으므로 3번 쉼터에서 5번 쉼터로 이동할 수 없다.

\end{problem}
