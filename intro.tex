\section*{참가자를 위한 도움말}

\subsection*{주의 사항}

\begin{itemize}
\item 대회 시간은 13:00부터 17:00까지입니다. 대회가 진행되는 동안 인터넷 검색 및 전자기기 사용 등을 하실 수 없습니다. 단, 아래의 문서에 한해 대회 진행 중에도 참고하실 수 있으며, 책과 노트 등을 가져오신 경우 역시 참고하실 수 있습니다.
\begin{itemize}
  \item STL documentation: http://www.sgi.com/tech/stl/
  \item JDK documentation: http://docs.oracle.com/javase/8/docs/api/
  \item C++ reference: http://en.cppreference.com/w/cpp
\end{itemize}
\item 대회는 Baekjoon Online Judge(https://www.acmicpc.net)를 이용하여 진행됩니다. 별도로 제공되는 계정 정보를 이용하여 로그인하신 뒤 코드 제출 및 결과 확인 등을 하실 수 있습니다.
\item 모든 입력은 \emph{표준 입력}으로 주어지며, 모든 출력은 \emph{표준 출력}으로 합니다.
\item 테스트 케이스가 존재하는 문제의 경우, 테스트 케이스에 대한 출력을 모아서 하실 필요 없이, 각 테스트 케이스를 처리할 때마다 출력해도 괜찮습니다.
\item \emph{중요!!} \emph{리턴 코드}와 \emph{표준 오류(standard error, stderr) 스트림 출력} 에 주의하십시오. 프로세스가 $0$ 이 아닌 리턴 코드를 되돌리는 경우나 \emph{표준 오류 스트림에 출력}을 하는 경우 ``\emph{런타임 에러}'' 를 받게 됩니다.
\item 문제에 대한 질의 사항은 대회 페이지의 질문 기능을 사용해 주시기 바랍니다. 이 때 대답해 드리기 어려운 질문에 대해서는 ``\emph{답변을 드릴 수 없습니다}'' 로 대답될 수 있으므로 유의하십시오.
\end{itemize}

\subsection*{채점 결과에 대하여}

\begin{description}
\item[맞았습니다!!] 제출하신 답안이 모든 테스트 데이터를 정해진 시간 안에 통과하여 정답으로 인정되었음을 의미합니다.
\item[컴파일 에러] 제출하신 답안 프로그램을 컴파일하는 도중 오류가 발생하였음을 의미합니다.
\item[런타임 에러] 제출하신 답안 프로그램을 실행하는 도중 프로세스가 비정상적으로 종료되었음을 의미합니다.
\item[시간 초과] 제출하신 답안 프로그램이 정해진 시간 안에 종료되지 않았음을 의미합니다.
\item[틀렸습니다] 제출하신 답안 프로그램이 테스트 데이터에 대해 생성한 출력이 출제자의 정답과 일치하지 않음을 의미합니다.
\item[출력 초과] 제출하신 답안 프로그램이 지나치게 많은 양의 출력물을 생성하여 강제로 종료되었음을 의미합니다.
\item[출력 형식이 잘못되었습니다] 제출하신 답안 프로그램이 정해진 출력 형식을 따르지 않았음을 의미합니다.
\end{description}

만약 여러 가지의 원인으로 인해 ``\emph{맞았습니다!!}'' 가 아닌 다른 결과를 얻으셨다면, 그 중 어떤 것도 결과가 될 수 있습니다.
예를 들어 답도 잘못되었고 출력 형식도 잘못된 코드를 제출하신 경우 대부분 ``\emph{출력 형식이 잘못되었습니다}'' 를 받으시게 되지만, 경우에 따라서 ``\emph{틀렸습니다}'' 를 받을 수도 있습니다.

\clearpage
